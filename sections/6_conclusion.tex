\chapter{Future Work and Conclusion}
\label{chap:Conclusion}

\section{Future Work}
There are two main pathways to continue the work outlined in this study; development and extension efforts for WoB itself and further study of simulation results of Melbourne, both of which would improve the capacity and viability for WoB to meaningfully contribute to the field of public transport modelling. 

%Forecasting/ measuring government policy changes - modifications to GTFS timetables, population amplification, network modification/ scenario planning, 

\subsection{Model Extensions}
While WoB is currently a functioning model and has achieved this project's aims and objectives, there are additional avenues for future work and improvement. These largely focus on improving the overall workflow of using WoB, streamlining the process for increased efficiency and ease of use. 

In its present version, WoB can be used for fast iteration of different scenarios using modified GTFS files. However, manually editing GTFS files to change the network layout or timetable is not trivial and adds complexity to the scenario testing process. For future scenario testing, addition of a patronage demand amplification toggle bar would enable users to easily simulate future demand growth by multiplying the number of agents in the simulation. To extend the modelling pipeline encapsulated by WoB, built-in UI visualisation tools to generate plots similar to those in the case study could be developed. A built-in GTFS editor, input amplification and built in plotting would vastly improve WoB's ability to be used as a holistic tool for rail transport scenario planning. Future project scenarios such as the SRL or Melbourne Airport Rail could be explored using amplified demand and GTFS editor WoB extensions. 

Transfer handling in the model can also be strengthened by accounting for platform transfers and the general costs associated with making transfers. WoB currently uses GTFS that does not contain platform information, and needs additional features to handle the platform layout within each station for a more granular understanding of transfer behaviours. The general cost utility function (\Cref{eqn:cost_utility}) could also be expanded to include a transfer cost for each transfer that is required, as well as costs of the transfer attributes such as waiting and walking times \cite{cascajoStatedPreferenceSurvey2017}.

The case study lays the foundation for how WoB can provide detailed insights into rail use trends in Melbourne. However, further investigation is encouraged to more concretely explore the applications WoB has for Melbourne. With the imminent opening of the Metro Tunnel project introducing new transfer opportunities, the transfer handling limitations posed by the existing Myki data manipulation processes will become more pronounced. Therefore, the creation of the input O-D data should consider these transfer opportunities to better represent passenger journeys. Stated or revealed preference surveying would enable the crowding cost functions to be better calibrated to capture Melbourne residents' preferences regarding crowding and transfers. 

Additionally, to further calibrate the crowding cost functions for application in Melbourne, stated or revealed preference analysis should be undertaken to ensure the modelling is sensitive to crowding perception in Melbourne. Similar analysis can also be completed to better understanding transfer preferences which may influence transfer behaviour. 

\section{Conclusion}
Who's on Board (WoB) is a free and open source model that has been developed to assign origin-destination data to a rail transport network to produce meaningful service-level utilisation data. WoB is designed to model any timetabled network, taking a generic GTFS dataset as input. With a focus on speed and simplicity, the model's ability to transform origin-destination demand data collected from smart-card systems into service-specific utilisation data has been demonstrated through application to a case study on the Melbourne metropolitan rail network. The model's implementation of the multi-criteria RAPTOR pathfinding algorithm, along with a novel static bag data structure is optimised for speed. This ensures the model is accessible to use on consumer hardware, and when modelling the Melbourne network can simulate 8 rounds of replanning in 30 seconds with only 2.5GB of memory. 

WoB has broad applications for rail transport modelling and has potential to reveal novel insights due to the accurate and granular outputs. Because WoB models each service down to the minute, it lends itself to interrogating subtle patterns which are not typically accessible. The different transfer patterns, such as between platforms and groups of origin and destination stations, that emerge from the various collective dynamics of individual agents can be analysed using WoB. As WoB ingests a specific timetable to determine service-level demand, the effects of custom timetable alterations can be simulated and the results analysed with immediate feedback. WoB is a novel approach to modelling rail utilisation that forms the foundation of efforts to bolster understanding of intricate flows of passengers through public transport networks. 
