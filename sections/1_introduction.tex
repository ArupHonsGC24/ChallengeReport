\chapter{Introduction}
As transport networks rapidly expand, the accurate simulation of the impacts of prospective transport management policies becomes increasingly critical. The ability to collect and analyse detailed utilisation data is fundamental to understanding how these complex networks behave, now and into the future. For utilisation data to be a key driver of policy decision making, novel modelling methods that are capable of transforming large modern datasets into meaningful insights need to be designed and tested \cite{ghofraniRecentApplicationsBig2018, welchBigDataPublic2019}. 

A primary use-case for these modelling technologies is evaluating the utility of proposed infrastructure projects in the context of precise demand patterns. While many tools exist for modelling road networks and automobile use, these don't commonly include adequate mechanisms for modelling public transport (PT) networks, which require different considerations \cite{cortesMicrosimulationFlexibleTransit2005}. PT-specific predictive modelling tools are needed to accurately assess the impact of changes to network operation, including infrastructure upgrades, timetable updates or fare policy changes.

This paper presents  "Who's on Board?" (WoB), a novel rail service demand model designed to transform large datasets collected from smart-card-based PT networks into meaningful service-level utilisation data. The WoB reference application with graphical user interface (GUI) and full source code are available on GitHub \cite{bensutherlandWhosBoardRail2024}.

To illustrate the potential uses for WoB the city of Melbourne in Australia is used as case study. Using data from Melbourne's smart-card system, Myki, the crowding, utilisation and transfer characteristics of Melbourne's metropolitan rail network are visualised and further analysed. 

\section{Transport Modelling}
Transport modelling allows new projects to be tested virtually, lowering the possibility of costly mistakes with real infrastructure. The general purpose of modelling is to represent a real world system to better understand that system. More specifically, a model can be used to analyse both existing and proposed systems in real or hypothetical scenarios. 

For transport service demand modelling, the purpose is to understand the interactions between service demand and supply provisioning, beyond what can be observed from raw data alone. Modelling demand at the individual vehicle level creates a far more detailed picture than using overall network capacity and averaged passenger entries as metrics of supply and demand. the ten or hundreds of billions of dollars, a "try besfore you buy" approach becomes increasingly crucial \cite[p.~12]{victorianparliamentarybudgetofficeSuburbanRailLoop2024}.

\subsection{Transport Modelling in Melbourne}
The city of Melbourne in Australia was used as a case study rail network to test the WoB model. Melbourne's public transport system collects detailed patronage data, including both boardings and alightings data and origin-destination data, and has a digitised timetable available in a common format, making it easier to model. These two aspects are common for many public transport systems worldwide, enabling this case study to function as a generalisable example for application of the WoB model. 

\subsection{Mode choice}
When examining general transport networks, there are many transport modes to consider. Automobile use makes up the vast majority of transport, due to the significant amount of investment in road infrastructure globally \cite{minerCarHarmGlobal2024, prieto-curielABCMobility2024}. Commuter data can be considered representative in the absence of data that provides better coverage to discuss mode share. For example, the 2021 Australian census found that in Victoria 85.9\% of commuters travel by car to work every day \cite{australianbureauofstatisticsCensusPopulationHousing2021}. Active and public transport are less commonly used, with trains used by 4.6\% of commuters daily, buses and trams together used by 2.4\%, cycling 1.1\% and walking 3.7\%. As the share of road transport increases, the negative consequences of car use, such as the health impacts of air and noise pollution, sedentary lifestyles, and the climate impacts of emissions will also intensify \cite{frankPathwaysBuiltEnvironment2019, macleodCommutingWorkPostpandemic2022}. Therefore without accessible and appealing public transport these trends will continue \cite{currieAlarmingTrendsGrowth2018, minerCarHarmGlobal2024}. As such, WoB focuses on modelling PT with the aim of improving PT outcomes.

The WoB model runs on a generic timetabled network and therefore, in theory, can model any public transport mode, or connected combination of modes. However, to receive an accurate output, the input data must also be robust. Because of the accessibility and quality of the case study rail data, WoB has been tuned to model rail networks, thus rail modelling is the focus of this research. 

\section{Scope}
\subsubsection{Aim}
To develop a general model for assigning passenger trips from origin-destination data to a rail transport network layout and timetable, producing meaningful service-level utilisation data. 

\subsubsection{Objectives}
\begin{SingleSpacing}
\begin{itemize}
    \item Build a service-level model of train utilisation which is fast, accurate and granular. 
    \item Build a simple user interface for the model that is easy to use. 
    \item Describe crowding cost functions for model implementation. 
    \item Showcase potential model use cases through visualisation of Melbourne train utilisation. 
\end{itemize}
\end{SingleSpacing}

\subsubsection{Key Areas of Focus}
Transport modelling is a broad discipline which attempts to address a wide range of use cases and problems. Given the multitude of ways any single problem can be approached and the solution implemented, there is no ultimate model that can be applied in every situation or solve every problem. Since there are always trade-offs or simplifying assumptions that must be made, different modelling software is specialised for different purposes. By comparing these models, a niche was identified where there are use cases and target users that are not accounted for by existing software \cite{horniMultiAgentTransportSimulation2016, adnanAgentbasedSimulationPlatforma, ZenithMultimodalTransport}. 

With a goal to understand PT utilisation, and support a wide range of end users such as engineers, government planners, advocacy groups and the general public, the following model characteristics have been identified as key areas of focus:

\begin{SingleSpacing}
\begin{itemize}
    \item Accessible
    \item Fast
    \item Granular
    \item Accurate
    \item Predictive
\end{itemize}
\end{SingleSpacing}

The majority of existing transport modelling software is intended for corporate or government use and requires a high level of technical knowledge. To engage a wider range of end users, there are no special knowledge requirements to run the model and it is free and openly accessible to anyone. The setup and operation of the model as described in this paper does not require understanding of the underlying implementation. In contrast to many existing modelling software that are expensive and closed source, the WoB model is free to access in its entirety on GitHub \cite{bensutherlandWhosBoardRail2024}.

Another key area of focus was the speed of the model for rapid feedback and iteration times for the end user. This is particularly useful in planning sessions with professionals and members of the public, as faster run times allows more scenarios to be simulated and altered in real time to generate dialogue and discussion around plans \cite{levinsonApplicationsAccess}. Comprehensive industry-focused models with a broad range of applications (including various modes and additional parameters) can be much more resource intensive to run and as a result, much slower.

Although speed can be bought with greater quantities of hardware with higher specifications, to account for end users from the general public, an ideal model should be able to run on most standard home computers, as opposed to specialised server infrastructure. Currently the average computer memory tends to be between 8GB to 16GB, so memory usage within this range without significantly slowed processing would enable people and groups without significant computational resources to access the model \cite{SpecificationsPersonalComputers}. 

Accurate and granular predictive modelling forms the additional components of the key focuses of the model. The ability to model service-level demand provides far greater detail than aggregated patronage counts, allowing more nuanced understanding of transport trends to emerge. Similarly, ensuring the accuracy of the model means the results can be trusted and used to inform decision making. This requires testing and iteration to verify the accuracy of results, which is much more rigorously carried out by proprietary software, compared to open source projects or one-off models. However, the model prioritises simplicity by optimising for a specific use case, instead of attempting to be applicable to more generalised problems. The more complex a system is, the more challenging it is to capture accurately. By focusing on a single mode, the rail network, rail specific modelling techniques can be applied, rather than adapting road-based techniques. This focused approach lays the foundation for future validation and testing of the model. 

\subsection{Target Users}
WoB's target users form a relatively more diverse range than that of large proprietary software, which tend to be aimed at corporations and engineers in particular. 

\subsubsection{Engineers}
The WoB model should be sufficiently accurate such that engineers can use it to inform or support their projects, such as station redevelopment, timetable changes or upgrades to rollingstock. While WoB is not expected to replace current models, it can complement existing tools by providing an additional perspective in transport modelling projects. 

\subsubsection{Government Planners}
Councils and government departments require predictive modelling capabilities to produce project feasibility studies and understand how policies or infrastructure projects could impact PT utilisation. WoB is free and open source allowing it to be easily accessed in the early stages of project planning before more costly modelling methods are required. The speed of the model also enables rapid adaptation and experimentation which may be costly if done through third-party consulting. 

\subsubsection{Political Advocacy and Special Interest Groups}
Many political advocacy and special interest groups do not have significant resources to spend on computing power or access to proprietary modelling software. Access to WoB would benefit these groups, allowing them to quantify the impact of their proposals. This can strengthen their position in advocating for policy changes such as service frequency upgrades if they can demonstrate the potential improvements. 

\subsubsection{General Public}
For the interested general public, the accessibility of WoB also allows these enthusiasts to gain a deeper understanding of how local or global transport networks behave. A built-in option to generate "random" patronage allows users to interact with the model, irrespective of whether they have access to detailed patronage data. 

\section{Outline}
The following chapter (\Cref{chap:Background}) describes the theoretical background of transport modelling in terms of approaches, simulations, and pathfinding algorithms. The technical details of the construction of WoB is presented in \Cref{chap:ModelConstruction}, outlining the high-level algorithm and the requisite sub-problems, before detailing the specific implemented solutions. \Cref{chap:ModelSpec} discusses the crowding cost functions which describe passenger perception of crowding to inform the allocation of demand to services. \Cref{chap:CaseStudy} demonstrates potential applications of WoB through a case study on the Melbourne metropolitan rail network. Finally, \Cref{chap:Conclusion} sets out potential pathways for future research and concluding remarks.

