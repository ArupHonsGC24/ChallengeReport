\chapter{Introduction}
[HOOK / Fact about good understanding/ model of PT vital for decision making, project seeks to address this problem]

Understanding how travellers use Melbourne's public transit network is a key requirement of effective service management policy. Utilisation data can be used to determine the provisioning of services, the collection of fares and plan novel infrastructure projects. Modelling technologies that can characterise both current demand and predict the future growth trajectories of a policy or project are an invaluable resource for determining how limited funds can be spent in a targeted fashion. These techniques range widely in scope and applicability, each with its own advantages and limitations.

\section{Motivation for Research}
In general, the purpose of modelling is to provide a representation of a real world system (to a variable degree of accuracy) such that a greater understanding of that system can be gained. More specifically, a model can be used to analyse both existing and proposed systems in either real or hypothetical scenarios. In the case of transport service demand modelling, the purpose is to understand the interactions between service demand and supply provisioning at a level deeper than what could be drawn from raw data alone. Going beyond simply using overall network capacity as a metric of supply and averaged passenger entries as demand and instead modelling demand at the individual vehicle level creates a much more detailed picture. This type of analysis can be useful across many scales, from updating the regular service timetable to informing the nature and design of infrastructure projects. With transport projects with budgets in the tens of billions of dollars \cite[p.~12]{victorianparliamentarybudgetofficeSuburbanRailLoop2024}, a "try before you buy" approach is a base requirement to ensure sound investment.

Melbourne's metropolitan rail services carried 157 million passengers and travelled over 25 million kilometres in the 2022-23 financial year, with an estimated running cost of \$1.4 billion over this period \cite[p.~47]{victoriandepartmentoftransportandplanningDepartmentTransportPlanning2023}. For the suburban rail demand to be adequately satisfied in the present and into the future, current demand needs to be comprehensively quantified such that future demand can be accurately forecast. Currently, the primary source of rail usage data is from smart-card (Myki) data, which in the best case can only record boarding and alighting numbers at each station, but will likely have issues with fare non-interaction. From this data we cannot directly infer how crowded trains might be, or how travellers make decisions on how to transfer between stations. For example, a current rail-only model used by the Department of Transport and Planning (DTP) to extrapolate demand from Myki boarding data does not take into account maximum vehicle capacities and therefore cannot directly model train crowding. Crowding causes shifts in service demand which leads to increased demand in later services. Such a model finds it difficult to produce a precise illustration of which train services are being used the most.

Modelling of crowding on public transport is a historically under-researched area compared to crowding on roads \cite{boumanPassengersCrowdingComplexity2017}. The DTP's more comprehensive Victorian Integrated Transport Model (VITM) takes these and many other factors into account over a multi-modal network and is therefore very complex and requires specialised training to use. However, the VITM's representation of public transport compared to automobile use is a simplification due to the trade-offs required in multi-modal modelling. The VITM and similar models used by the industry are not widely accessible from a technical or availability standpoint, and therefore their role in shaping government policy can be obscured.

\section{Research Gap}
This thesis proposes a new model that is specifically designed to examine rail use and crowding. The proposed model is designed to be both fast and accurate, while being accessible without specialised knowledge. The goal of the model is to use patronage data collected from smart card systems (such as Myki) to calculate crowding on public transport at the individual service level. This model could be used to inform timetable alterations and infrastructure improvements for more efficient network operation and a more comfortable passenger experience. This contribution to the field of transport modelling combines existing algorithms and techniques together to create a novel modelling approach.

\subsection{Fast}
For the types of problems that are addressed by service demand modelling, such as predicting the effect of an update to the regular timetable, quick iteration and feedback are required to converge on an optimal solution. It is therefore important to maintain a focus on speed.

Although speed can be bought with faster and greater quantities of hardware, to ensure the model is accessible it must be optimised to be workable on a reasonably standard home computer (a computer many end-users might possess, as opposed to server infrastructure). Maintaining sensible memory and time limits allows the model to be used by people and groups without significant computational resources.

\subsection{Accurate}
Because there are the significant differences between modelling public transport networks and modelling personal automobiles on road networks, to avoid inaccuracy or additional complexity it is best to design a model solely for one type of network, rather than attempting to adapt a model from one type of network to use for the other type. This model will be specifically focused on determining demand for rail services, and will not use road-based modelling techniques.

The model should accurately quantify or predict the crowding of each service, i.e. the number of people on each service relative to the vehicle's capacity. The overall framework approach of this research is designed to be accurate, but due to data accessibility limitations the accuracy of the model was not tested.

\subsection{Accessible}
To ensure that the model finds widespread use, it must be accessible in keeping with the following criteria:
\subsubsection{No Special Knowledge Requirements}
Many commercially available modelling tools require specialised technical knowledge to set up and run. While knowledge when available is not an insurmountable obstacle, making sure the knowledge requirement is as low as possible can greatly improve the software's accessibility. Setting up and running the model as implemented shouldn't require knowledge of the internals, although modification and extension workloads will naturally have a higher bar.
\subsubsection{Free and Open Source}
As public transport modelling has largely been the endeavour of large organisations, the established modelling software is generally expensive and closed source (with the one notable exception of the "Multi-Agent Transport Simulation" (MATSim) software \cite{horniMultiAgentTransportSimulation2016}, briefly discussed later). To minimise the barrier to entry, all code developed for this thesis is freely available on GitHub \cite{bensutherlandArupHonsGC24Trainute2024}.

Aside from being cost free, the freedom to modify the code modules to different or more specific circumstances than have been accounted for allows the overall framework to find wider applicability.
\subsubsection{Web-based Visualisation}
The model results can be exported, viewed and shared in the web-based visualisation framework. Once a model has been run using the optimised executable, the results can be viewed and interpreted by anyone with a web browser.
\subsection{Target Users}
\subsubsection{Engineers}
The model should be useful enough that engineers can employ the results to inform their projects, and accurate enough that the model output can be relied upon.
\subsubsection{Government Planners}
Government departments at the local or state level that require predictive modelling of how their policies affect rail utilisation can also use the proposed model.
\subsubsection{Political Advocacy and Special Interest Groups}
Many political advocacy and special interest groups may not have significant resources to spend on computing power. These groups will benefit from accessing an accurate predictive model to justify their policy agenda, such as service frequency updates.
\subsubsection{General Public}
For the interest of the general public the model should be technically and computationally accessible, so that enthusiasts can gain their own deeper understanding of how their local transport network behaves.

\section{Outline and Methodology}
Following the computer science convention, the structure of this report mirrors the structure and flow of the project. The next section (\Cref{chap:Background}) looks at the theoretical background of the problem by considering relevant modelling techniques, pathfinding algorithms, data formats and visualisation methods from the literature. From this background, various techniques were attempted until a cohesive framework design was created. The overall approach of this simulation framework is presented in \Cref{chap:ModelConstruction} by laying out the high-level algorithm and the requisite subproblems, before detailing the specific solutions. The two major subproblems that were solved are laid out in the following two chapters. \Cref{chap:ModelSpec} addresses the problem of pathfinding, and the implementation of two interchangeable algorithms that can be used in the simulation. An overview of the techniques used to visualise the model output is described in \Cref{chap:CaseStudy}, along with a link to a live demo. Finally, \Cref{chap:Conclusion} sets out potential pathways for future research and some concluding remarks.
